\documentclass[letterpaper, 12pt]{article}
\usepackage{cite, amsmath, amsthm, amssymb, amsfonts, bbm, bm, listings, enumitem, geometry, titlesec, array, makecell, caption, subcaption, graphicx, float, mathtools}
\usepackage[breaklinks]{hyperref}
\geometry{margin=1in}

\title{Proposal}
\author{Yuanhao JIANG}
\date{}

\begin{document}
\maketitle

\paragraph{Title:}
Extending Score-Based Diffusion Models for 3D Structure Generation.
\paragraph{Supervisor:}
Professor George Deligiannidis.
\paragraph{Brief description:}
Score-based diffusion models \cite{song2020score,song2021maximum} use SDEs to gradually add noise to data and then reverse the process to generate structured samples, requiring accurate score function estimation via neural networks trained with score matching.

Despite the success of it, the application of score-based diffusion models to 3D structure generation remains underexplored. 3D data pose unique challenges, including maintaining geometric consistency and handling non-Euclidean data. In applications such as 3D shape modeling or protein structure prediction, constraints such as smooth surfaces and structural integrity must be satisfied, necessitating methods that can generate constrained and structured data effectively. Building on the foundational work in SGMs \cite{song2020score,song2021maximum} and potentially intersection with constrained dynamics \cite{huang2022riemannian,chung2022improving,de2022riemannian}, this project seeks to extend score-based diffusion to 3D data, addressing these challenges with following objectives
\begin{enumerate}
    \item Baseline Exploration for Standard Score-Based Diffusion.
    \item Extension to 3D Data: Adapt score-based diffusion models for 3D structure generation, focusing on point clouds or voxel grids.
    \item Manifold-Aware Generation: Investigate techniques for constraining the diffusion process to 3D manifolds, ensuring geometric consistency and structural validity.
\end{enumerate}
\paragraph{Prerequisite courses/knowledge:}
Stochastic processes, differential equations, numerical solvers for SDEs, deep learning frameworks (PyTorch).
\paragraph{Data availability:}
A custom constrained dataset will be generated for trial, with extensions to opensource datasets \href{http://mocap.cs.cmu.edu}{CMU Mocap} or \href{http://vision.imar.ro/human3.6m/description.php}{Human3.6M} later.
\paragraph{Computing:}
The project requires SRF GPUs at the later experimental stage for potential SDE simulations and training neural networks in score-based diffusion models.

% \section{Methodology}
% The plan is to first modify the network architecture to operate on 3D data, for example, for point clouds, use Graph Neural Networks (GNNs) or PointNet based architectures to capture the spatial relationships between points. For voxel grids, employ CNN based networks to process volumetric data. Adapt the noise injection and score estimation mechanisms to align with the dimensionality and spatial characteristics of 3D data.

% To incorporate constraints into the generative process, several approaches can be explored, focusing on constrained dynamics, manifold-awareness, and penalty-based methods
% \begin{enumerate}
%     \item Project out-of-manifold samples back onto the constraint surface at each timestep.
%     \item Add penalty term \cite{chung2022improving}.
%     \item Riemannian Score-based Generative Modelling \cite{de2022riemannian}.
% \end{enumerate}

\bibliographystyle{plain}
\bibliography{references}

\end{document}